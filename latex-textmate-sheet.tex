\makeatletter\def\input@path{{applekeys/}}\makeatother
%-----------------------------------------------------------------------
% Document description and required packages
%
\documentclass[10pt,a4paper,landscape]{article}
\usepackage{multicol}
\usepackage{calc}
\usepackage{ifthen}
\usepackage[landscape]{geometry} 
\usepackage{graphics}
\usepackage{hyperref}
\usepackage{pdfsync}
\usepackage{dot2texi}
%-----------------------------------------------------------------------

%-----------------------------------------------------------------------
% Change default fonts for the document
%
\usepackage[utf8]{inputenc}
\usepackage[T1]{fontenc}
\usepackage[scaled]{helvet}
\renewcommand*\familydefault{\sfdefault}
\renewcommand*\ttdefault{txtt}
%-----------------------------------------------------------------------



%-----------------------------------------------------------------------
% This sets page margins to .5 inch if using letter paper, and to 1cm
% if using A4 paper. (This probably isn't strictly necessary.)
% If using another size paper, use default 1cm margins.
%
\ifthenelse{\lengthtest { \paperwidth = 11in}}
	{ \geometry{top=.5in,left=.5in,right=.5in,bottom=.5in} }
	{\ifthenelse{ \lengthtest{ \paperwidth = 297mm}}
		{\geometry{top=1cm,left=1cm,right=1cm,bottom=1cm} }
		{\geometry{top=1cm,left=1cm,right=1cm,bottom=1cm} }
	}

% Turn off header and footer
\pagestyle{empty}

% Redefine section commands to use less space%
\makeatletter
\renewcommand{\section}{\@startsection{section}{1}{0mm}%
                                {-1ex plus -.5ex minus -.2ex}%
                                {0.5ex plus .2ex}%x
                                {\normalfont\large\bfseries}}
\renewcommand{\subsection}{\@startsection{subsection}{2}{0mm}%
                                {-1explus -.5ex minus -.2ex}%
                                {0.5ex plus .2ex}%
                                {\normalfont\small\bfseries}}
\renewcommand{\subsubsection}{\@startsection{subsubsection}{3}{0mm}%
                                {-1ex plus -.5ex minus -.2ex}%
                                {1ex plus .2ex}%
                                {\normalfont\footnotesize\bfseries}}

%-----------------------------------------------------------------------
% Defines a command to easily insert keyboard modifier from pdf images
%
\newcommand{\key}[1]{
	\scalebox{0.012}{\includegraphics{{#1}.pdf}}
}
%-----------------------------------------------------------------------

\makeatother

%-----------------------------------------------------------------------
% Define BibTeX command
%
\def\BibTeX{{\rm B\kern-.05em{\sc i\kern-.025em b}\kern-.08em
    T\kern-.1667em\lower.7ex\hbox{E}\kern-.125emX}}

% Don't print section numbers
\setcounter{secnumdepth}{0}

\setlength{\parindent}{0pt}
\setlength{\parskip}{0pt plus 0.5ex}
%-----------------------------------------------------------------------

%-----------------------------------------------------------------------
% Here we go! 
%
\begin{document}

\raggedright
\footnotesize
\begin{multicols}{4}

% multicol parameters
% These lengths are set only within the two main columns
%\setlength{\columnseprule}{0.25pt}
\setlength{\premulticols}{1pt}
\setlength{\postmulticols}{1pt}
\setlength{\multicolsep}{1pt}
\setlength{\columnsep}{2pt}
%-----------------------------------------------------------------------

%-----------------------------------------------------------------------
% Document title
%
\begin{center}
     \Large{\textbf{TextMate and \LaTeX\\Cheat Sheet}} \\
\end{center}

\key{escape}Escape (esc), \key{tab}Tab, \key{revtab}Reverse Tab (\key{shift}\key{tab}),\\
\key{shift}Shift, \texttt{fn} Function (fn), \key{control}Control (ctrl),\\
\key{option}Option (alt), \key{command}Command (cmd), \\
\key{return}Carriage Return, \key{delete}Delete.

\texttt{|}: alternatives in shortcuts

\emph{Shortcuts in italic refer to custom shortcuts.}

%----------------------------------------------------------------------- 

%-----------------------------------------------------------------------
% TextMate generic shortcuts
%
\section{TextMate} % (fold)
\label{sec:textmate}

\subsection{Generic shortcuts} % (fold)
\label{sub:generic_shortcuts}
\begin{tabular}{@{}ll@{}}
	\key{escape}						& Auto-completion\\
	\key{command}\texttt{\{}			& Environment based on current word\\
	\key{option}\key{command}\texttt{.}	& Close current environment\\
	\key{command}\texttt{\}}			& Command based on current word\\
	\key{command}\texttt{/}				& Comment/Uncomment line/selection 
\end{tabular}


\emph{Environments and Commands can be generated from many keywords (as shown
here). More can be added by going to \texttt{Bundles $\rightarrow$ LaTex
$\rightarrow$ Edit Configuration File}.}
% subsection generic_shortcuts (end)

\subsection{Bundles} % (fold)
\label{sub:bundles}
\begin{tabular}{@{}ll@{}}
	\key{control}\key{option}\key{shift}\texttt{L}	& LaTeX\\
	\key{control}\key{option}\key{shift}\texttt{B}	& LaTeX Beamer\\
	\key{control}\key{option}\key{shift}\texttt{M}	& LaTeX Memoir
\end{tabular}
% subsection bundles (end)

\subsection{Project management} % (fold)
\label{sub:project_management}
\begin{tabular}{@{}ll@{}}
	\key{control}\key{shift}\texttt{A} 	& SubVersioN control menu\\
	\key{control}\texttt{L}				& Open project master file\\ 
\end{tabular}
% subsection project_management (end)

% section textmate (end)
%-----------------------------------------------------------------------

%-----------------------------------------------------------------------
% Compilation and output
%
\section{Compilation and output} % (fold)
\label{sec:compilation_and_output}
\begin{tabular}{@{}ll@{}}
	\key{control}\key{shift}\texttt{V}					& Verify LaTeX code (chktex)\\
	\key{command}\texttt{R}								& Compile and view (PDF)\\
	\key{control}\key{option}\key{command}\texttt{O}	& Show in PDF viewer\\
	\key{control}\key{command}\texttt{W}				& Watch document\\
	\key{control}\key{option}\key{delete}				& Delete aux files
\end{tabular}
% section compilation_and_output (end)
%-----------------------------------------------------------------------

%-----------------------------------------------------------------------
% Document structure
%
\section{Document structure} % (fold)
\label{sec:document_structure}

\begin{tabular}{@{}ll@{}}
	\key{shift}\key{command}\texttt{O}	& Show Document Outline
\end{tabular}

\subsection{Document Header} % (fold)
\label{sub:document_header}
\begin{tabular}{@{}ll@{}}
	\texttt{dc}\key{command}\texttt{\}}		& Document class (article by default)\\
	\texttt{nc}\key{command}\texttt{\}}		& New command\\
	\texttt{rnc}\key{command}\texttt{\}}	& Renew command\\
	\texttt{usep}\key{command}\texttt{\}}	& Use Package\\
	\texttt{geo}\key{command}\texttt{\}}	& Use Geometry Package\\
	\emph{\texttt{inc}\key{command}\texttt{\}}}	& \emph{Include a tex file in document}\\
\end{tabular}
% subsection document_header (end)

\subsection{Sectionning} % (fold)
\label{sub:sectionning}

\begin{tabular}{@{}ll@{}}
	\texttt{part}\key{tab}		& New part\\
	\texttt{cha}\key{tab}		& New chapter\\
	\texttt{sec}\key{tab}		& New section\\
	\texttt{sub}\key{tab}		& New subsection\\
	\texttt{subs}\key{tab}		& New sub-subsection\\
	\texttt{par}\key{tab}		& New paragraph\\
	\texttt{subp}\key{tab}		& New sub-paragraph\\
	\texttt{subp}\key{tab}		& New sub-paragraph\\
	\key{control}\texttt{*}		& Toggle starred
\end{tabular}

\emph{New structures will be created with associated label and folding attributes.\\
Label name is automatically derived from the structure name.}
% subsection sectionning (end)



\subsection{Environments} % (fold)
\label{sub:environments}
\begin{tabular}{@{}ll@{}}
	\texttt{begin}\key{tab}							& Create environnement ($\backslash$begin\{\}\ldots$\backslash$end\{\})\\
	\key{control}\key{shift}\key{command}\texttt{W}	& Wrap current selection in environment\\
	\key{command}\texttt{\{}						& Environment based on current word\\
	\key{option}\key{command}\texttt{.}				& Environment closer
\end{tabular}
% subsection environments (end)

\subsection{Commands} % (fold)
\label{sub:commands}

\begin{tabular}{@{}ll@{}}
	\key{control}\key{shift}\texttt{W}	& Create command\\
	\key{command}\texttt{\}}			& Command based on current word\\
	\key{command}\key{escape}			& Command completion
\end{tabular}

\emph{The Create command will generate and highlight the mostly used command.}
% subsection commands (end)`

\subsection{Lists} % (fold)
\label{ssub:lists}

\begin{tabular}{@{}ll@{}}
	\texttt{enum}\key{tab} \textit{or} 	\texttt{en|enum}\key{command}\texttt{\{}	& New enumerate\\
	\texttt{item}\key{tab} \textit{or} 	\texttt{it|ietm}\key{command}\texttt{\{}	& New itemize\\
	\texttt{desc}\key{tab} \textit{or} 	\texttt{desc}\key{command}\texttt{\{}		& New description\\
	\texttt{fn}\key{return}															& New item\\
	\texttt{itd}\key{tab}															& New item with description\\
	\key{control}\key{shift}\texttt{L}												& Itemize lines in selection
\end{tabular}

\emph{Item with description is required under the description environment.\\
New item shortcut will automatically add a new line.\\
Itemize lines in selection displays a popup that lets you chose between
Left/Right and Itemize lines.}
% subsection lists (end)

% section document_structure (end)
%-----------------------------------------------------------------------

%-----------------------------------------------------------------------
% Formatting
%
\section{Formatting} % (fold)
\label{sec:formatting}
\begin{tabular}{@{}ll@{}}
	\key{command}\texttt{B} \textit{or} \texttt{bf}\key{command}\texttt{\}}	& Bold\\
	\key{command}\texttt{I}	\textit{or} \texttt{e}\key{command}\texttt{\}}	& Italic (emphasize)\\
	\texttt{it}\key{command}\texttt{\}}										& Italic\\
	\key{command}\texttt{U}				& Underline\\
	\key{command}\texttt{K}				& Typewriter\\
	\key{option}\key{command}\texttt{K}	& Verbatim\\
	\key{shift}\key{command}\texttt{K} \textit{or} \texttt{sc}\key{command}\texttt{\}}	& Small caps
\end{tabular}
% section formatting (end)
%-----------------------------------------------------------------------

%-----------------------------------------------------------------------
% Content and References
%
\section{Content and References} % (fold)
\label{sec:content_and_references}

\subsection{Tables} % (fold)
\label{sub:tables}
\begin{tabular}{@{}ll@{}}
	\texttt{tab}\key{tab}							& New Tabular environment\\
	\key{control}\key{shift}\key{command}\texttt{T}	& Create Table Wizard \\
	\texttt{tbl|table}\key{command}\texttt{\}}		& Centered table with caption and label\\	
	\key{control}\key{shift}\key{command}\texttt{T}	& Convert Selection to Table\\
	\texttt{fn}\key{return}							& Add row\\
	\key{control}\key{option}$\rightarrow$			& Add column\\
	\key{control}\key{option}$\leftarrow$			& Remove column\\
	\key{control}\key{tab}			 				& Next cell\\
	\key{control}\key{revtab}		 				& Previous cell\\
	\texttt{hf}\key{command}\texttt{\}}				& Horizontal line (hfill)\\
	\texttt{vf}\key{command}\texttt{\}}				& Vertical line (vfill)\\
	\key{control}\texttt{\&}						& Reformat
\end{tabular}

\emph{The Convert Selection to Table uses a multi-line tabulation-separated
selection.}
% subsection tables (end)

\subsection{Other Labeled Environments} % (fold)
\label{sub:other_labeled_environments}
\begin{tabular}{@{}ll@{}}
	\texttt{fig|figure}\key{command}\texttt{\{}			& Figure\\
	\texttt{pic|picture}\key{command}\texttt{\{}		& Picture\\
	\texttt{lst|lstlisting}\key{command}\texttt{\{}		& Listings\\
	\texttt{def|definition}\key{command}\texttt{\{}		& Definition\\
	\texttt{pro|proposition}\key{command}\texttt{\{}	& Proposition\\
	\texttt{thm|theorem}\key{command}\texttt{\{}		& Theorem\\
	\texttt{lem|lemma}\key{command}\texttt{\{}			& Lemma\\
	\texttt{cor|corrolary}\key{command}\texttt{\{}		& Corrolary\\
	\texttt{pf|proof}\key{command}\texttt{\{}			& Proof
\end{tabular}
% subsection other_labeled_environments (end)

\subsection{Cross-references} % (fold)
\label{sub:references}
\begin{tabular}{@{}ll@{}}
	\texttt{figure}\key{tab}			& New reference to Figure\\
	\texttt{listing}\key{tab}			& New reference to Listing\\
	\texttt{page}\key{tab}				& New reference to Page\\
	\texttt{section}\key{tab}			& New reference to Section\\
	\texttt{table}\key{tab}				& New reference to Table\\
	\key{option}\key{escape}			& New cite from Bibtex\\
	\key{option}\key{escape}			& New reference from keyword\\
\end{tabular}
% subsection references (end)

% section content_and_references (end)
%-----------------------------------------------------------------------

%-----------------------------------------------------------------------
% Mathematics
%
\section{Mathematics} % (fold)
\label{sec:mathematics}

\subsection{Maths environments} % (fold)
\label{sub:maths_environments}
\begin{tabular}{@{}ll@{}}
	\texttt{\$}																	& Display inline Maths \\
	\texttt{\$\$}\key{tab}														& Display Maths \\
	\key{control}\key{shift}\texttt{M}											& Maths mode\\
	\texttt{eq}\key{tab} \textit{or} 	\texttt{eq|eqn}\key{command}\texttt{\{}	& Equation environment\\
	\texttt{eq*|eqn*}\key{command}\texttt{\{}									& Unnumbered Equation environment\\
	\texttt{eqa|eqnarray}\key{command}\texttt{\{}								& Equation array\\
	\texttt{eqa*|eqnarray*}\key{command}\texttt{\{}								& Unnumbered Equation array\\
\end{tabular}

\emph{For inline Maths, typing a single \texttt{\$} sign will cause TextMate to
autocomplete with the closing sign.\\
The Maths mode command displays a popup menu that lets you chose between
Display Maths and Maths mode.}
% subsection maths_environments (end)

\subsection{Maths structures} % (fold)
\label{sub:maths_structures}
\begin{tabular}{@{}ll@{}}
	\texttt{ali}\key{tab}			& Align(ed)\\
	\texttt{gat}\key{tab}			& Gather(ed)\\
	\texttt{spl}\key{tab}			& Split\\
	\texttt{mat}\key{tab}			& Matrix\\
	\texttt{cas}\key{tab}			& Cases
\end{tabular}
% subsection maths_structures (end)

\subsection{Maths expressions} % (fold)
\label{sub:maths_expressions}
\begin{tabular}{@{}ll@{}}
	\texttt{fr|frac}\key{command}\texttt{\}}		& Fraction\\
	\texttt{l|lim}\key{command}\texttt{\}}			& Limit\\
	\texttt{s|sum}\key{command}\texttt{\}}			& Sum\\
	\texttt{p|prod}\key{command}\texttt{\}}			& Product\\
\end{tabular}
% subsection maths_expressions (end)

% section mathematics (end)
%-----------------------------------------------------------------------

%-----------------------------------------------------------------------
% Beamer
%
\section{Beamer} % (fold)
\label{sec:beamer}

\begin{tabular}{@{}ll@{}}
	\texttt{fr}|\texttt{frame}\key{command}\texttt{\{}		& New frame\\
	\texttt{col}|\texttt{column}\key{command}\texttt{\{}	& New column\\
	\texttt{bl}|\texttt{block}\key{command}\texttt{\{}		& New block\\
	\key{command}\texttt{<}									& Add beamer overlay\\
\end{tabular}

\emph{New beamer frame is created with default attribute \texttt{t} which aligns
content to the top. \texttt{frametitle} option is automatically added.}
% section beamer (end)
%-----------------------------------------------------------------------

%-----------------------------------------------------------------------
% Footer of the document
\rule{0.3\linewidth}{0.25pt}
\scriptsize

Copyright \copyright\ 2009 Guillaume-Jean Herbiet\\
Revision: 0.4, Date: \today \\
\url{http://www.herbiet.net}
%-----------------------------------------------------------------------

%-----------------------------------------------------------------------
% That's all folks!
\end{multicols}
\end{document}
%-----------------------------------------------------------------------
